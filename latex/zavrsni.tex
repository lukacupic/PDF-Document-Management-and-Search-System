\documentclass[times, utf8, zavrsni]{fer}
\usepackage{booktabs}

\begin{document}

% TODO: Navedite broj rada.
\thesisnumber{5672}

% TODO: Navedite naslov rada.
\title{Sustav za upravljanje i pretraživanje baze PDF dokumenata}

% TODO: Navedite vaše ime i prezime.
\author{Luka Čupić}

\maketitle

% Ispis stranice s napomenom o umetanju izvornika rada. Uklonite naredbu \izvornik ako želite izbaciti tu stranicu.
\izvornik

% Dodavanje zahvale ili prazne stranice. Ako ne želite dodati zahvalu, naredbu ostavite radi prazne stranice.
\zahvala{}

\tableofcontents

\chapter{Uvod}
Područje analize i pretraživanja teksta neizbježno je u današnjem svijetu tehnologije. Od internetskih tražilica koje pretražuju enormne količine podataka baziranih na zadanom upitu, osobnih pomoćnika na pametnim mobitelima koji procesiraju izgovorene riječi pa sve do analize i prepoznavanja \textit{spam} elektroničkih poruka.
Kratki osvrt na ove te mnoge druge primjene ukazuju na nepobitnu činjenicu da je pretraživanje teksta...

\chapter{Pregled područja}
Korištenje računala za povrat informacija (engl. \textit{information retrieval}) datira čak do 1940-ih godina, daleko prije komercijalizacije računala. Očito je da je to problem...


\chapter{Let's dive right into it...}

\section{Prikaz dokumenata}
Prije \textit{poniranja u dubine}, objasnimo prvo što u kontekstu analize i pretraživanja predstavlja dokument. Neformalno dokument možemo definirati kao kolekciju riječi. Ovakva kolekcija riječi ne mora nužno biti skup, pošto dokument može imati više ponavljanja istih riječi; ova će činjenica doći do izražaja u \textbf{poglavlju X}. Umjesto toga, pretpostavljamo da se dokument sastoji od  tzv. \textit{vreće riječi} (engl. \textit{bag of words}) kod koje nam nije bitna semantika samog dokumenta, pa čak niti poredak riječi, već je bitna samo činjenica da se riječi pojavljuju u dokumentu, odnosno učestalost njihovog pojavljivanja.
Ovakav model često je korišten u području procesiranja prirodnog jezika te povrata informacija kako iz dokumenata tako iz drugih izvora tekstualnih informacija.

Da bi se dokumenti mogli predstaviti u obliku vreća riječi, potrebno je odrediti vokabular—skup svih riječi koje se nalaze u svim dokumentima promatrane kolekcije dokumenata (u daljnjem tekstu: zbirka). Iz ovako zadanog vokabulara ćemo ponajprije, ukloniti sve zaustavne riječi (engl. \textit{stop words})—riječi koje su učestale u nekom jeziku te su stoga nebitne za sam postupak analize teksta. Primjeri nekih zaustavnih riječi u hrvatskom jeziku su: \textit{aha}, \textit{nešto}, \textit{okolo}, \textit{zaboga}. Osim zaustavnih riječi, dodatna obrada teksta može se obaviti tzv. \textit{stemanjem} (engl. \textit{stemming}). Ova metoda ima zadaću svesti riječi na njihov kanonski oblik. Drugim riječima, nebitno je je li riječ napisana u jednini ili množini ili pak u kojem je padežu; bitan je samo kanonski oblik riječi. Na primjer, riječi poput \textit{spavao} i \textit{spavati} svesti će na \textit{spavanje}.
Nakon stvaranja vokabulara te predobrade dokumenata (izbacivanje zaustavnih riječi, stemanje) možemo krenuti s predstavljanjem dokumenata. Radi praktičnosti, najčešća metoda predstavljanja dokumenata jest uz pomoć vektora.
Najjednostavnija metoda vektorskog predstavljanja dokumenata jest binarna: za svaku riječ iz vokabulara naprosto provjerimo nalazi li se u danom dokumentu te ukoliko se nalazi, odgovarajuća komponenta vektora (indeks riječi u vokabularu) biti će 1, a u suprotnom 0. Nadograđujući se na prethodnu metodu, dolazimo do frekvencijskog prikaza vektora. Umjesto obične binarne reprezentacije u kojoj pamtimo samo nalazi li se riječ u dokumentu ili ne, u frekvencijskom prikazu pamtimo i koliko se puta dotična riječ pojavljuje u dokumentu (komponente vektora zapravo su frekvencija (tj. broj) pojavljivanja određene riječi u dokumentu). Naposlijetku dolazimo i do najčešće metode vektorskog prikaza dokumenata—TF-IDF (engl. \textit{term frequency–inverse document frequency}). Ova metoda zasniva se na dvije intuitivne pretpostavke:
\begin{itemize}
\item Riječ je važnija za semantiku dokumenta što se češće u njemu pojavljuje (TF komponenta)
\item Riječ je manje važna za semantiku dokumenta što se češće pojavljuje u drugim dokumentima (IDF komponenta)
\end{itemize}
TF i IDF dakle predstavljaju dvije komponente vektora kojima ćemo predstavljati dokumente. Prva komponenta je već spomenuta, frekvencija pojavljivanja riječi \textit{w} u dokumentu \textit{d} ($f_\textit{w, d}$) dok je druga komponenta obrnuta frekvencija pojavljivanja riječi u cijeloj zbirci.
Za riječ \textit{w} i dokument \textit{d}, TF i IDF komponene računaju se na sljedeći način:
\begin{equation}
{\displaystyle \mathrm {tf} (t,d)=f_{t,d}}
\end{equation}
\begin{equation}
{\displaystyle \mathrm {idf} (t,D)=\log {\frac {N}{|\{d\in D:t\in d\}|}}}
\end{equation}
Formula za TF komponentu je intuitivna i trivijalna. Formula za IDF komponentu zahtjeva kratki osvrt: riječ će biti bitnija za neki dokument što se rijeđe pojavljuje u drugim dokumentima. Ovo vidimo u formuli kao omjer ukupnog broja dokumenata (veličine zbirke) \textit{N} te broja dokumenata koji sadrže gledanu riječ. Što je riječ više sadržana u ostalim dokumentima, omjer se smanjuje te riječ postaje manje bitna za neki dokument. Naposlijetku, cijeli se omjer logaritamski skalira kako bi se u smanjio utjecaj velikog broja dokumenata i/ili malog broja dokumenata koji sadrže određenu riječ, na vrijednost IDF-a.

\section{Semantička sličnost dokumenata}
Nakon izgrađene vektorske reprezentacije dokumenata, sljedeći korak jest samo uspoređivanje dokumenata. Uspoređivanje se može ostvariti na dva različita načina: uspoređivanje korisničkog unosa (engl. \textit{user input}, \textit{query}) sa zbirkom dokumenata ili uspoređivanje dokumenata međusobno.

\subsection{Semantička sličnost dokumenata}
Lorem ipsum dolor sit amet, consectetur adipiscing elit, sed do 
eiusmod tempor incididunt ut labore et dolore magna aliqua. Ut 
enim ad minim veniam, quis nostrud exercitation ullamco...

\chapter{Programsko rješenje}
Cilj ovog rada jest istražiti već ranije spomenute metode analize i pretraživanja teksta.	Kako je implementacija programskog rješenja specifična za dokumente tipa PDF, takve dokumente prvo treba preprocesirati kako bi bili spremni za obradu.
Programske potpora kao implementacija problema ovog završnog rada napisana je u programskom jeziku Java. Razlog ovakvog odabira leži u tome što je Java popularan objektno orijentirani jezik po čemu je idealan za rješavanje problema poput DOCUMENT RETREIVAL-a zbog svoje nativne podrške apstraktnih kolekcija podataka, podrške raznih biblioteka i sl.
U svrhu preprocesiranja PDF dokumenata, koristi se Apache PDFBox koji omogućava brzo i jednostavno izvlačenje teksta iz PDF dokumenata.
Kao algoritam za stemanje riječi koristi se poznati 'Portland Stemming Algorithm' čije su implementacije javno dostupne u većini popularnijih programskih jezika, pa tako i u Javi.

\chapter{Zaključak}
Zaključak.

\bibliography{literatura}
\bibliographystyle{fer}

\begin{sazetak}
Sažetak na hrvatskom jeziku.

\kljucnerijeci{Ključne riječi, odvojene zarezima.}
\end{sazetak}

% TODO: Navedite naslov na engleskom jeziku.
\engtitle{Title}
\begin{abstract}
Abstract.

\keywords{Keywords.}
\end{abstract}

\end{document}
